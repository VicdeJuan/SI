\documentclass{apuntes}

\title{Sistemas informáticos (P1,P2)}
\author{Guillermo Julián, Víctor de Juan}
\date{Octubre - 2014}

\begin{document}

\pagestyle{plain}
\maketitle

\tableofcontents
\newpage

\chapter{Práctica 1}

Los ficheros relevantes para esta práctica se encuentran en la carpeta \textit{P1}. Abriendo en el navegador \textit{index.xml} se puede ver el diseño inicial de la página. En cuanto supimos que no había entrega, empezamos a trabajar en la práctica 2, terminando lo que faltaba en los ficheros correspondientes de la práctica 2.

\section{Funcionalidades propias}
Lo interesante de esta práctica es el XML, el XSL y la definición del DTD asociado. El resto de funcionalidades han sido utilizadas también en la Práctica 2 y se comentarán en el siguiente capítulo.


\subsection{XML}

\paragraph{Historial: }
El historial con XML y mostrado con XSL no lo llegamos a implementar (porque la práctica 1 no se iba a entregar), así que en las películas es donde utilizamos el xml (posteriormente, en la práctica 2 sí utilizamos XML para el historial).

\paragraph{Películas: }
El listado de películas se encuentra en \textit{index.xml}. No destaca especialmente por su completitud, pero ese problema lo solucionamos utilizando \textit{imdb} en la práctica 2. 

\subsection{XSL} 

\paragraph{Películas: }
El \textit{XSL} que da estilo al \textit{index.xml} viene asociado al principio con:
\begin{verbatim}
<xml-stylesheet type="text/xsl" href="style/moviecatalog.xsl"?>
\end{verbatim}

El XSL define cómo se muestran las peliculas en el HTML y el estilo de las clases se encuentra en \textit{main.css} (de la práctica 1). Todas las clases utilizadas por el XSL para películas son de la forma \textit{movie-...}


\paragraph{Historial: }
Al no haber implementado el historial, no llegamos a definir un XSL asociado al XML del historial.

\subsection{DTD}

El DTD que define el correcto formato del XML viene asociado al princpio con:

\begin{verbatim}
<!DOCTYPE catalog SYSTEM "data/catalog.dtd">
\end{verbatim}


\chapter{Práctica 2}


\section{Estructura}

\subsection{Mapa de navgación}

Debido a la simplicidad de navegación de nuestra Web hemos prescindido de incluir un esquema. Sólo tenemos 3 páginas: \begin{itemize}
\item \textbf{\textit{register.php}}: Crear una cuenta nueva.
\item \textbf{\textit{history.php}}: Muestra el historial de compras del usuario loggeado.
\item \textbf{\textit{index.php}}: Vista de películas y búsqueda.
\end{itemize}

A la página del historial se accede haciendo clic en el nombre después de haber entrado con la cuenta de usuario correspondiente. 

A la página de registro se accede pulsando el botón de \textit{login} y después en el enlace \textit{¿No tienes cuenta todavía?} que aparece en el diálogo. 


\subsection{Ficheros importantes}

\subsubsection{XML}

Se encuentran en formato XML el historial de cada usuario (en un archivo llamado \textit{history.xml} dentro de la carpeta de cada usuario) y las películas (en \textit{data/movies.xml})


El historial de cada usuario se escribe y lee con PHP utilizando el objeto \textit{SimpleXML}, funcionalidad de la que hablaremos más adelante.



\subsubsection{CSS}

Todos los estilos de todas las clases se encuentran dentro de la carpeta \textit{style}.

Lo más destacable del CSS la posición de la cabecera y del pie de página, que siempre aparecen, aunque hagamos más pequeña la pantalla del navegador. Estos estilos están definidos en el estilo de las etiquetas \textit{footer} y \textit{header} de HTML5.

\subsubsection{PHP}

Los ficheros PHP se encuentran divididos en 2 carpetas, \textit{php} y \textit{api}. En la primera están las funciones \textit{de librería}, por así decirlo, las que actúan de interfaz frente a las estructuras de datos definidas.

En la carpeta \textit{api} se encuentras las interfaces REST implementadas con PHP y que exponen a clientes (de momento, sólo el cliente JavaScript de la propia página web) usando las ``funciones librería'' de la carpeta \textit{php}.

\label{CriterioPHP}

\paragraph{Carrito}

Distinguimos 2 ámbitos: el lado del servidor y el del cliente. \label{Carrito}

\subparagraph{Servidor}


Definimos un array en la variable de sesión para no perder la información al recargar la página. La información no persiste entre sesiones, ya que no almacenamos la información en ningún fichero, simplemente en la información de sesión.

Está implementado con peticiones http (GET, POST y DELETE según sea la acción realizada por le usuario). La comunicación se realiza formateada utilizando JSON.

\subparagraph{Cliente}

No existe un único controlador de Angular encargado de la gestión del carrito, ya que tiene que ser accesible desde varios controladores, por lo que definimos una variable \textit{cart} con unos métodos asociados (\textit{find, add, get, ...}) de tal manera que el resto de controladores pueden actualizar el carrito.

\paragraph{Películas}

Como en el resto de módulos, vamos a comentar primero el lado del servidor y después el del cliente.

\subparagraph{Servidor}
En el servidor encontramos 2 ficheros php (\textit{php/movies.php} y \textit{api/movies.php}), el primero expone tres funciones para gestionar la lista de películas (buscar una película determinada, buscar todas la películas y obtener las películas en un cierto rango), mientras que el segundo se encarga de responder a las peticiones HTTP sobre las películas.

\subparagraph{Cliente}
El controlador de Angular encargado de mostrar las películas es \textit{movieListController}, que se encarga de mostrarlas, de añadir al carrito las compras y de los filtros, que mencionaremos más adelante, ya que hemos creado una directiva en Angular para realizarlo.


\paragraph{Historial} Los archivos php utilizados en el historial se describen más adelante, en la sección de funcionalidades.

\paragraph{Salir}

Simplemente eliminamos las variables de identifación de \textit{\$\_SESSION} y redirigimos a \textit{index.php}. Mantenemos, sin embargo, el carrito de compra.


\section{Funcionalidades}

\subsection{Registro - Login}

\paragraph{Registro}

El registro de un usuario nuevo tiene 2 partes, el formulario (se encuentra en \textit{register.php}) y el procesado del registro (se encuentra en REVISAR: \textit{php/login\_register.php}).

Para acceder a un nuevo registro, abrimos el diálogo de login y nos vamos al enlace \textit{¿No tienes cuenta todavía?}. 

La validación de los campos se realiza con las etiquetas de HTML5 o rellenando el campo \textit{pattern} que tienen todas las etiquetas, salvo la contraseña, que para medir la fortaleza utilizamos \textit{JavaScript}. Hablamos de esto más adelante (Angular: \ref{Angular}, Fortaleza de la contraseña: \ref{PasStrength})

\paragraph{Login}

Tiene 2 partes importantes:

\begin{itemize}

\item El lado del servidor (en el fichero REVISAR: \textit{php/login\_register.php}) que lee del fichero y comprueba las contraseñas utilizando md5.

\item El lado del cliente, que utiliza JavaScript, en concreto la librería de Angular para mostrar o no el cuadro del login. Todo lo relativo a esta funcionalidad se encuentra en el controlador \textit{loginSubmitController}, en el fichero \textit{js/controller.js}. Utilizando las directivas \textit{ng-show}, \textit{ng-click} y \textit{ng-init} resulta sencillo mostrar el cuadro de login o no. 

El cuadro de login se ha implementado con un \textit{div} en posición absoluta. Para más información del estilo, consultar las clases \textit{login-div} y \textit{login-form}.

\end{itemize}


\subsection{Búsqueda}
El filtrado de películas se realiza combinando JavaScript y PHP. 

\begin{itemize}
\item El código PHP se encuentra dividido en 2 fichero: \textit{api/movies.php} y \textit{php/movies.php} (siguiendo el criterio comentado anteriormente (\ref{CriterioPHP}). Utilizamos JSON para la comunicación de datos entre cliente y servidor.

\item El código JS se encuentra en \textit{js/controller.js} y el controlador de Angular que realiza esta funciones es \textit{movieListController}.

Desde Angular se realizan peticiones http al servidor para obtener las películas por páginas y no cargar datos innecesarios.

El filtrado se hace en cliente con JavaScript. Para ello hemos usado una directiva personalizada para implementar la interfaz de los filtros, y un filtro de Angular para hacer la búsqueda en sí.

La directiva es \textit{filter} y está definida en \textit{js/filterDirective.js}. Se trata de código reutilizable que genera automáticamente el HTML para manejar filtros en base a ciertos parámetros dados (la plantilla HTML se encuentra en \textit{base/filter.html}).

Cada elemento de filtro actualiza una propiedad del objeto \textit{search} del controlador de películas, de tal forma que la búsqueda se hace en el preciso instante en el que el suuario cambia los filtros.

En el caso de que al filtrar las películas se muestren menos que el tamaño de página configurado por el usuario, el controlador procede a llamar al servidor para obtener la siguiente página de películas. Esa página se obtiene sin filtrar desde el servidor, ya que creemos que es más eficiente guardarnos las películas que no se vayan a mostrar por si el usuario las pide más tarde que hacer peticiones extra cada vez que cambien los filtros.
\end{itemize}



\subsection{Historial}

El historial tiene 2 funciones, añadir películas al historial o mostrar las películas compradas. 

El php necesario también se encuentra dividido según en criterio ya mencionado (\ref{CriterioPHP}).

\begin{itemize}
\item \textbf{Añadir} películas al historial: desde JS (en concreto, desde el controlador \textit{headerController}) se realiza una petición POST al servidor con los datos de las películas formateados en JSON. El servidor recibe este POST y añade un nuevo pedido al historial, con los id's de las películas y las cantidades compradas. 

\item \textbf{Mostrar} películas del historial: esta funcionalidad se encuentra en otro fichero, en \textit{history.php}, en el directorio raíz, ya que elegimos mostrar el historial en otra página aparte. Aquí, el controlador REVISAR realiza una petición GET, obteniendo un array de arrays estructurado de la siguiente manera:

\begin{verbatim}
Objeto recibido => \{
		0 => pedido1,
		1 => pedido2,
		...
	\}
\end{verbatim}
Donde pedido se define como 

\begin{verbatim}pedido => \{
	0 => movie,
	1 => movie,
	...
	n => date
\}
\end{verbatim}

A su vez, movie tiene los campos:

\begin{verbatim}
movie => \{
	'id' => int,
	'quantity' => int,
	`price' => int
	'date' => fecha de la compra
\}
\end{verbatim}

REVISAR: completar

\end{itemize}

\subsection{Otros}

\subsubsection{Fortaleza de la contraseña}
\label{PasStrength}

La comprobación se realiza en JS, en concreto, en el controlador \textit{registerController}, que define una variable utilizada con la directiva \textit{ng-show} para mostrar la información de la fortaleza de la contraseña conforme la vayamos rellenando. El único requisito para contraseña inválida es una longitud menor de 5 caracteres.

\subsubsection{Carrito}

Ya ha sido comentado en la sección de los ficheros de PHP  (\ref{Carrito}).

\subsubsection{Usuarios online}

Esta funcionalidad se encuentra en el pie de la página. El controlador de Angular \textit{footerController} realiza una petición al servidor (un GET a \textit{api/webinfo.php}) y actualiza las variables que son mostradas en el html.

\section{Fuentes externas}

\subsection{Angular}
\label{Angular}

Angular es una librería de JavaScript nueva que permite separar la lógica de la vista de la del modelo que ésta representa. Así, en JavaScript sólo tenemos que actualizar las variables y será Angular el que se encargue de actualizar las propiedades enlazadas correspondientes en el HTML.

Siguiendo con esta lógica y utilizando las directivas que ofrece Angular, se consigue una página dinámica en la que el JS \textbf{solamente} actualiza variables, no modifica campos del html. Esta librería hace posible no utilizar \textit{document.getElementBy...}, lo cual facilita mucho la tarea al desacoplar el ``qué mostramos'' del ``cómo lo mostramos'' y acaba repercutiendo en un código mucho más limpio, claro y modular.

Además, permite definir tus propias directivas para después poder ser reutilizadas, lo que permite un código HTML muy sencillo.


Al utilizar Angular, hemos podido prescindir de llamadas directas a XmlHttpRequest (aunque por supuesto hemos implementado funcionalidades asíncronas) y de la librería jQuery.

\subsection{IMDB}

Para obtener los datos de las películas hemos realizado un pequeño script en Python (\textit{imdb\_downloader.py}) que utiliza la librería \href{https://github.com/richardasaurus/imdb-pie}{ImdbPie} para descargar varios \textit{top} de películas o películas concretas a través del título. Una vez descargados los datos y guardada la imagen de cada película en el directorio correspondiente, se escribe toda la información en un fichero XML para poder ser usado después por nuestro servidor.

\subsection{Referencias}

\begin{itemize}
	\item \href{http://www.bennadel.com/blog/2658-using-scope-watch-to-watch-functions-in-angularjs.htm}{Cómo usar \$scope.\$watch para observar el valor de funciones en Angular}.
	\item \href{http://nathanleclaire.com/blog/2014/04/19/5-angularjs-antipatterns-and-pitfalls/}{Angular antipatterns and pitfalls}.
	\item \href{https://github.com/angular/angular.js/wiki/Understanding-Scopes}{La documentación de Angular sobre ámbitos}.
	\item \href{http://stackoverflow.com/questions/17706847/not-sure-how-to-hide-a-div-when-clicking-outside-of-the-div}{Cómo ocultar un div al hacer clic fuera de él.}
	\item \href{http://www.colorcombos.com/color-schemes/22/ColorCombo22.html}{La paleta de colores usada en nuestra web.}
	\item \href{https://docs.angularjs.org/guide/directive}{Documentación de Angular sobre diretivas personalizadas.}
\end{itemize}

\end{document}