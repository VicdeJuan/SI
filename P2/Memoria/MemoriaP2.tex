\documentclass{apuntes}

\title{Sistemas informáticos (P1,P2)}
\author{Guillermo Julián, Víctor de Juan}
\date{Octubre - 2014}

\begin{document}

\pagestyle{plain}
\maketitle

\tableofcontents
\newpage

\chapter{Práctica 1}

Los ficheros relevantes para esta práctica se encuentran en la carpeta \textit{P1}. Abriendo \textit{index.xml} se puede ver el diseño inicial de la página. En cuanto supimos que no había entrega, empezamos a trabajar en la Práctica 2, terminando lo que faltaba en los ficheros correspondientes de la práctica 2.

\section{Funcionalidades propias}
Lo interesante de esta práctica es el Xml, el xsl y la definición del dtd asociado. El resto de funcionalidades han sido utilizadas también en la Práctica 2 y se comentarán en el siguiente capítulo.


\subsection{XML}

\paragraph{Historial: }
El historial con XML y mostrado con XSL no lo llegamos a implementar (porque la práctica 1 no se iba a entregar), asique en las películas es donde utilizamos el xml (posteriormente, en la práctica 2 si utilizamos XML para el historial).

\paragraph{Películas: }
El listado de películas se encuentra en \textit{index.xml}. No destaca especialmente por su completitud, pero ese problema lo solucionamos utilizando \textit{imdb} en la práctica 2. 

\subsection{XSL} 

\paragraph{Películas: }
El \textit{XSL} que da estilo al \textit{index.xml} viene asociado al principio con:
\begin{verbatim}
<xml-stylesheet type="text/xsl" href="style/moviecatalog.xsl"?>}
\end{verbatim}

El XSL define cómo se muestran las peliculas en el HTML y el estilo de las clases se encuentra en \textit{main.css} (de la práctica 1). Todas las clases utilizadas por el XSL para películas son de la forma \textit{movie-...}


\paragraph{Historial: }
Al no haber implementado el historial, no llegamos a definir un XSL asociado al XML del historial.

\subsection{DTD}

El dtd que define el correcto formato del xml viene asociado al princpio con:

\begin{verbatim}
\textit{<!DOCTYPE note SYSTEM "data/catalog.dtd">}
\end{verbatim}


\chapter{Práctica 2}


\section{Estructura}

\subsection{Mapa de navgación}

\subsection{Ficheros importantes}

\subsubsection{XML}

Se encuentran en formato XML el historial de cada usuario (en un archivo llamado \textit{users/email@usuario/history.xml} dentro de la carpeta de cada usuario) y las películas (en \textit{data/movies.xml})


El historial de cada usuario se escribe y lee con PHP utilizando el objeto \textit{SimpleXML}, funcionalidad de la que hablaremos más adelante.



\subsubsection{CSS}

Todos los estilos de todas las clases se encuentran dentro de la carpeta \textit{style}.

Lo más destacable del CSS la posición de la cabecera y del pie de página, que siempre aparecen, aunque hagamos más pequeña la pantalla del navegador. Estos estilos están definidos en el estilo de las etiquetas \textit{footer} y \textit{header} de HTML5.

\subsubsection{PHP}

Los ficheros PHP se encuentran divididos en 2 carpetas, \textit{api} y \textit{php}. 

\label{CriterioPHP}
REVISAR: En \textit{api} se encuentran ficheros php que definen .............., por el contrario, en \textit{php} encontramos los ficheros que implementan ...............

\paragraph{Carrito}

\paragraph{Películas}

\paragraph{Historial}

\paragraph{Salir}


\section{Funcionalidades}

\subsection{Registro - Login}

\paragraph{Registro}

El registro de un usuario nuevo tiene 2 partes, el formulario (se encuentra en \textit{register.php}) y el procesado del registro (se encuentra en REVISAR: \textit{php/login\_register.php}).

Para acceder a un nuevo registro, abrimos el diálogo de login y nos vamos al enlace \textit{¿No tienes cuenta todavía?}. 

La validación de los campos se realiza con las etiquetas de HTML5 o rellenando el campo \textit{pattern} que tienen todas las etiquetas, salvo la contraseña, que para medir la fortaleza utilizamos \textit{JavaScript}. Hablamos de esto más adelante (Angular: \ref{Angular}, Fortaleza de la contraseña: \ref{PasStrength})

\paragraph{Login}

Tiene 2 partes importantes:

\begin{itemize}

\item El lado del servidor (en el fichero REVISAR: \textit{php/login\_register.php}) que lee del fichero y comprueba las contraseñas utilizando md5.

\item El lado del cliente, que utiliza JavaScript, en concreto la librería de Angular para mostrar o no el cuadro del login. Todo lo relativo a esta funcionalidad se encuentra en el controlador \textit{loginSubmitController}, en el fichero \textit{js/controller.js}. Utilizando las directivas \textit{ng-show}, \textit{ng-click} y \textit{ng-init} resulta sencillo mostrar el cuadro de login o no. 

El cuadro de login se ha implementado con un \textit{div} en posición absoluta. Para más información del estilo, consultar las clases \textit{login-div} y \textit{login-form}.

\end{itemize}


\subsection{Búsqueda}
El filtrado de películas se realiza combinando JavaScript y PHP. 

\begin{itemize}
\item El código PHP se encuentra dividido en 2 fichero: \textit{api/movies.php} y \textit{php/movies.php} (siguiendo el criterio comentado anteriormente (\ref{CriterioPHP}). Utilizamos JSON para la comunicación de datos entre cliente y servidor.

\item El código JS se encuentra en \textit{js/controller.js} y el controlador de Angular que realiza esta funciones es \textit{movieListController}.

Desde Angular se realizan peticiones http al servidor para obtener el listado de películsa (dinámicamente, en función de cuántas películas estemos mostrando en cada momento). 

\end{itemize}



\subsection{Historial}

El historial tiene 2 funciones, añadir películas al historial o mostrar las películas compradas. 

El php necesario también se encuentra dividido según en criterio ya mencionado (\ref{CriterioPHP}).

\begin{itemize}
\item \textbf{Añadir} pelíuclas al historial: desde JS (en concreto, desde el controlador \textit{headerController}) se realiza una petición POST al servidor con los datos de las películas formateados en JSON. El servidor recibe este POST y añade un nuevo pedido al historial, con los id's de las películas y las cantidades compradas. 

\item \textbf{Mostrar} películas del historial: esta funcionalidad se encuentra en otro fichero, en \textit{history.php}, en el directorio raíz, ya que elegimos mostrar el historial en otra página aparte. Aquí, el controlador REVISAR realiza una petición GET, obteniendo un array de arrays estructurado de la siguiente manera:

Objeto recibido => \{
		0 => pedido1,
		1 => pedido2,
		...
	\}
	
Donde pedido => \{
	0 => movie,
	1 => movie,
	...
\}

A su vez, movie tiene los campos:

movie => \{
	'id' => int,
	'quantity' => int,
	'date' => fecha de la compra
\}

REVISAR: completar

\end{itemize}

\subsection{Otros}

\subsubsection{Fortaleza de la contraseña}
\label{PasStrength}

La comprobación se realiza en JS, en concreto, en el controlador \textit{registerController}, que define una variable utilizada con la directiva \textit{ng-show} para mostrar la información de la fortaleza de la contraseña conforme la vayamos rellenando. El único requisito para contraseña inválida es una longitud menor de 5 caracteres.

REVISAR: completar el código.

\subsubsection{Carrito}

REVISAR: completar

\section{Fuentes externas}

\subsection{Angular}
\label{Angular}

Angular es una librería de JS nueva, que permite separar la vista, del modelo, del controlador, permitiendo escribir en HTML variables de JS ( {{variable }}), que cuando cambia de valor, se actualiza el HTML con el nuevo valor. Siguiendo con esta lógica y utilizando las directivas ng-show por ejemplo, se consigue una página dinámica en la que el JS \textbf{solamente} actualiza variables, no modifica campos del html. Esta librería hace posible no utilizar \textit{document.getElementBy...}, lo cual facilita mucho la tarea.

Además, permite definir tus propias directivas para después poder ser reutilizadas, lo que permite un código HTML muy sencillo.

\subsection{imdb}

REVISAR: El XML de las películas lo generamos con un script en python (\textit{imdb\_downloader.py}) utilizando la API \textit{imdb}. 


\subsection{Referencias}

REVISAR: copy-paste al entregar.

\end{document}