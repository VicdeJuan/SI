\documentclass[nochap]{apuntes}

\title{Sistemas informáticos - práctica 3}
\author{Guillermo Julián, Víctor de Juan}
\date{Noviembre- 2014}

\begin{document}

\pagestyle{plain}
\maketitle

\tableofcontents
\newpage

\section{Diseño de la base de datos}
\subsection{Entidad - relación}

\subsection{Modificaciones sobre la base de datos}

\section{Consultas, procedimientos almacenados y triggers}
\subsection{setItemPrice}
\subsection{setOrderAmount}
\subsection{getTopVentas}
\subsection{getTopMonth}
\subsection{updOrders}
\subsection{updInventory}

\section{Modificaciones sobre el HTML, JS y PHP}
En general no ha habido que realizar grandes cambios, debido a que intentamos ser lo más modulares posibles.

\subsection{Registrarse y Login}

Solo ha sido necesario cambiar el fichero \textit{/php/login_register.php} en el que se realizan las comprobaciones de login y registrarse. Utilizamos un \textit{PDO} para poder preparar las consultas, ya que siempre serán del mismo tipo, cambiando algún argumento, para lo que utilizamos el método \textit{bindParams}.

\subsection{Películas}
\subsection{Historial}

\end{document}