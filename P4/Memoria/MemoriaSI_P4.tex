\documentclass{article}
\usepackage[T1]{fontenc}
\usepackage[utf8x]{inputenc}
\usepackage[spanish]{babel}
\usepackage{graphicx}
\usepackage{hyperref}
\usepackage[left=3cm, right=2cm, top=3cm, bottom=2cm]{geometry}

\title{Práctica 4 SI}
\date{\today}
\author{Guillermo Julián Moreno \and Víctor de Juan Sanz}
\newcommand{\eqdef}{\stackrel{{\scriptsize\rm def}}{=}}

\begin{document}
\maketitle

\section{Optimización}

\subsection{Estudio del impacto de un índice}

\subsection{Estudio del impacto de preparar sentencias SQL}

\subsection{Estudio del impacto de cambiar la forma de realizar una consulta}

\subsection{Estudio del impacto de la generación de estadísticas}

\section{Transacciones y \textit{deadlocks}}

\subsection{Transacciones}

\subsection{Estudio de bloqueos y \textit{deadlocks}}

\section{Seguridad}

\subsection{Acceso indebido a un sitio web}

Para acceder al sistema como el usuario \textit{gatsby}, simplement introducimos ese nombre en el campo de usuario y, en contraseña, introducimos \texttt{' OR 1=1; {-}-}. Aprovechamos así el fallo de programación en \textit{xLoginInjection.php}, de tal forma que el código realizará la consulta

\begin{center}
\begin{verbatim}
SELECT * FROM customers WHERE username='gatsby' AND password='' OR 1 = 1; --
\end{verbatim}
\end{center}

Es decir, siempre será válida y sacará al menos un registro. El código PHP dará por válido el \textit{login} y entraremos sin saber la contraseña.

La misma técnica se puede usar para entrar sin saber el nombre de usuario: al usar la cláusula \texttt{OR}, no es siquiera necesario que el nombre de usuario esté en la base de datos, ya que \texttt{1=1} es una condición que siempre es verdadera para cualquier registro.

Para evitar este tipo de ataques, se debe escapar las cadenas que vengan desde el cliente con \texttt{pg\_escape\_string} o bien usar la clase \textit{PDO} con \texttt{prepare}, que automáticamente realiza el escape necesario. Así impediremos que en las cadenas que introduce el usuario haya caracteres que puedan modificar el comportamiento de nuestra consulta SQL.

\subsection{Acceso indebido a información}



\end{document}
